%% copyrightpage
\thispagestyle{empty}
\begingroup
\footnotesize
\parindent 0pt
\parskip \baselineskip
\textcopyright{} 2022 Κωνσταντίνος Χωριανόπουλος\\
Creative Commons (CC BY-NC-ND 3.0 GR).

Αναφορά Δημιουργού-Μη Εμπορική Χρήση-Όχι Παράγωγα Έργα 3.0 Ελλάδα. 

Φωτογραφία εξωφύλλου, MIT Lincoln Lab, Ivan Sutherland, Sketchpad.

\vfill

Αυτό το φυσικό βιβλίο σε μέγεθος Α4 με μαλακό εξώφυλλο βασίζεται σε αρχικό προσχέδιο και σελιδοποίηση του συγγραφέα και παράγεται αποκλειστικά για ιδιωτική διανομή και αρχειοθέτηση στην Εθνική Βιβλιοθήκη της Ελλάδος. Τα δικαιώματα για την διανομή στο κοινό της τελικής έκδοσης με τίτλο "Κατασκευή Συστημάτων Διάδρασης" και ηλεκτρονική παραγωγή από την δράση ΚΑΛΛΙΠΟΣ έχουν μεταφερθεί στον ΣΕΑΒ και στον ΕΛΚΕ ΕΜΠ. Το τελικό ηλεκτρονικό βιβλίο πραγματοποιήθηκε με την χρηματοδότηση του έργου ΚΑΛΛΙΠΟΣ+ και είναι ελεύθερα διαθέσιμο με άδεια χρήσης Creative Commons (CC BY-NC-SA 3.0 GR) στο αποθετήριο της δράσης ΚΑΛΛΙΠΟΣ: \texttt{https://repository.kallipos.gr} 

\vfill

\begin{center}
978-618-82423-7-1
\end{center}

\begin{center}
\begin{tabular}{ll}
Πρώτη έκδοση: & Ιανουάριος 2016 \\
Δεύτερη έκδοση, με μικρές διορθώσεις & Σεπτέμβριος 2019 \\
Τρίτη έκδοση, με σημαντικές επεκτάσεις & Δεκέμβριος 2022 
\end{tabular}
\end{center}

\vfill

Γλωσσική επιμέλεια \\
\hspace*{2em} Πρώτη έκδοση, Σοφία Ντούλου \\
\hspace*{2em} Δεύτερη έκδοση, Μαρία Σδρόλια \\

\vfill

Χωριανόπουλος, Κωνσταντίνος\\
\hspace*{2em} Ο Προγραμματισμός της Διάδρασης \\
\hspace*{2em} Αυτο-έκδοση \\
\hspace*{2em} 242 σελ. \hspace*{2em} Α4. \\
\hspace*{2em} Περιλαμβάνει εικόνες και βιοβλιογραφικές αναφορές. \\
\hspace*{2em} ISBN 978-618-82423-7-1 \\
\hspace*{2em} 1. Ηλεκτρονικοί υπολογιστές \hspace*{2em} I. Διάδραση Ανθρώπου-Υπολογιστή 


\vfill

Κωνσταντίνος Χωριανόπουλος, \\
Αθήνα, Ελλάδα \\
\texttt{https://pibook.epidro.me}

%%%%{\LARGE\plogo}
\vspace*{2\baselineskip}


\endgroup
